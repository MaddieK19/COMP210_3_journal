% Please do not change the document class
\documentclass{scrartcl}

% Please do not change these packages
\usepackage[hidelinks]{hyperref}
\usepackage[none]{hyphenat}
\usepackage{setspace}
\doublespace

% You may add additional packages here
\usepackage{amsmath}
\usepackage{graphicx} 
\graphicspath{ {Figures/} }

% Please include a clear, concise, and descriptive title
\title{COMP210 Research Journal}

% Please do not change the subtitle
\subtitle{COMP210}

% Please put your student number in the author field
\author{1507866}

\begin{document}
	
\maketitle
	
\abstract{This essay will look at a number of factors effecting Virtual Reality (VR), it's development and its uses.}

\section{Introduction}
The intention of this essay is to look at many factors effecting Virtual Reality (VR). The areas addressed will be; VR interfaces and evaluating them, controls in VR and  human factors in VR. 
	
\section{VR Interfaces and Evaluation}
Mentzelopoulos \textit{et al} researched hardware interfaces for VR \cite{Mentzelopoulos}.  
One important factor in a VR interface is latency. High latency can cause issues in VR that effects the player's experience \cite{Mentzelopoulos, Meehan}. Which can have a negative impact on the player's sense of presence in a virtual environment \cite{Meehan}. 
 
\bigskip 
 
Meehan conducted a study on virtual environments designed to cause stress. The results showed that participants with a lower latency had larger increases in nausea and an increased heart rate \cite{Meehan}.Thus,  as the intent of the virtual environment was to cause stress the results suggested lower latency was more successful.

\bigskip

Another factor is tracking both the player's position and input \cite{Mentzelopoulos}. Mentzelopoulos \textit{et al} said that a VR interface should be spontaneous and need no explanation. Their study involved participants using either XBox or Razer motion controllers for input. The study found the game that used motion controllers to be more fun. This could be due to the game not necessarily the controller itself.  However, McArthur's study of the Wiimote and  Wiimote accessories found that the shape of the controller effected the player's accuracy and error rates.  Suggesting that the controller can affect the player's experience.

\bigskip  

A potential issue with Mentzelopoulos' study is that there were only 18 participants. Also all participants were either computing or computer science students which may have effected the results \cite{Mentzelopoulos}. However, they stated that the results of the study were unclear and that more research is necessary.

\bigskip 
 
A heuristics analysis can be used to evaluate a VR interface.  The analysis can identify weaknesses in the interface. The results can then be used to improve the interface \cite{Nielsen, Pinelle}.


Pinelle \textit{et al} created a series of heuristics specific to video game usability. These heuristics could be used to analyse a VR interface. However, Pinelle's heuristics were not designed for use in VR games.  Therefore, some heuristics may not apply or need to be altered.  Sutcliffe presented a method based on Nielson's heuristics to analyse virtual environments \cite{sutcliffe2004heuristic}.

\section{Player and Avatar Relationship}

Won \textit{et al} conducted experiments on the psychological and physiological effects of  VR avatars on the player \cite{won2015homuncular}. 
There are many previous studies that look at avatar embodiment and the fact that people can identify with avatars that differ from their own body \cite{Groen, Latoschik}.
The results of Won \textit{et al}'s experiments showed that players could adapt to avatars that worked differently to their own body. Also they suggested that intuitive controls could enable faster adaptation and greater success. There findings match that of the Protues effect which says that people will subconsciously change their behaviour to match an avatar in a virtual environment \cite{won2015homuncular, yee2007proteus}.


\section{VR Controls}
The Frustration --- Aggression model says that aggression is caused by a person being blocked from reaching their goals \cite{dollard1939frustration}.  This can be applied to games as some factor can prevent the player from obtaining their in game goal.  Przybylski \textit{et al}'s work showed that it is not necessarily violence in video games that causes player aggression and frustration. They suggest that it is instead competence impeding controls that cause aggression \cite{przybylski, przybylski2010motivational}. These competence impeding controls block that player from performing their desired in game action causing frustration. 


Kovarova and Maros researched the use of smart phones as an input device in VR games \cite{Kovarova}. They suggested that the traditional keyboard and mouse input for video games can be restrictive in a VR game as they are not intuitive in a virtual environment. Intuitive controls improve the VR experience and give the players a higher success rate at the tasks they want to accomplish \cite{Meehan, won2015homuncular}. Kovarova and Maros focused on the hardware of a smart phone that could be useful in a VR controller such as accelerometers. They also look at using a smart phone to make controls more intuitive \cite{Kovarova}.
A potential issue is that Kovarova and Maros do not look at issues such as latency.  Latency issues can break a player's sense of presence in the game. Therefore, any potential latency issues with using the smart phone could affect the player's VR experience \cite{Meehan}.

Bauer \textit{et al} conducted a study on using smart phones for display interactions \cite{Bauer}. They found that the small screen size had a negative effect. For 2D task participants managed to complete the given task even with little prior experience with a smart phone. However, using a smart phone to solve problems in 3D took longer and required more practice. VR is a 3D environment which suggests users may have some issues with using a smart phone as a controller and may take time to adapt to it.

\section{Human factors in VR}
Stanney \textit{et al} researched the effects of human factors on VR and its uses \cite{stanney}.  They say VR could be used in a larger variety of fields. However, for it to be widely used in fields such as medicine and engineering there are human factors that have to be considered. Greenleaf claims that early adoption of VR is likely to be in the games industry \cite{Greenleaf}. However, after that it will have a large variety of uses in medicine such as training, treatment and diagnosis \cite{Greenleaf}.  For VR to reach its full potential a number of human factors need to be researched. The areas Stanney looked at are; human performance efficiency in virtual environment, health and safety issues and potential social implications of virtual reality technology. 

Won \textit{et al} suggested players have a higher success rate when controls are intuitive \cite{won2015homuncular}. 
Similarly, Stanney says virtual environments should minimise how much the player has to learn to use the virtual environment. If the player cannot navigate the virtual environment their performance cannot be maximized which reduces VRs usefulness \cite{stanney}.  

Stanney also says that player variation should be taken into account. This can be physiological differences such as interpupillary distance or psychological differences such as different cognitive styles \cite{stanney}. Barfield found that differences in players can affect their sense of presence \cite{barfield1993sense}. Therefore, player differences should be taken into account when designing a virtual environment.

\section{Conclusion}
In conclusion there are many factors to consider when designing and using VR. Most of these relate to the player as issues such as latency and unintuitive controls can affect the player's presence.   Heuristics analysis such as Pinelle \textit{et al}'s and Sutcliffe's heuristics can be used to find many issues and improve the experience for the player.
	
\bibliographystyle{ieeetr}
\bibliography{comp210_3}
	
\end{document}
