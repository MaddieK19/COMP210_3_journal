% Please do not change the document class
\documentclass{scrartcl}

% Please do not change these packages
\usepackage[hidelinks]{hyperref}
\usepackage[none]{hyphenat}
\usepackage{setspace}
\doublespace

% You may add additional packages here
\usepackage{amsmath}
\usepackage{graphicx} 
\graphicspath{ {Figures/} }

% Please include a clear, concise, and descriptive title
\title{COMP 210 Research Journal}

% Please do not change the subtitle
\subtitle{COMP210}

% Please put your student number in the author field
\author{1507866}

\begin{document}
	
\maketitle
	
\abstract{}
	
\section{VR Interfaces and Evaluation}
Mentzelopoulos \textit{et al} researched hardware interfaces for VR \cite{Mentzelopoulos}. They said that a VR interface should be spontaneous and allow the player to understand it with no explanation.
One important factor that should be considered in a VR system is low latency \cite{Mentzelopoulos, Meehan}. 
 Meehan says that latency is and important factor in VR. Meehan describes high latency as an effect that can hinder or break the player's sense of presence in a virtual environment. Their study showed that in virtual environments designed to cause stress participants with low latencies had larger increases in heart rate and nausea \cite{Meehan}.
Another factor is tracking the player's position and monitoring their input \cite{Mentzelopoulos}. McArthur found that Wiimote accessories effected the player's accuracy and error rate \cite{McArthur}.
Mentzelopoulos \textit{et al}'s study involved participants playing with either XBox controllers or Razer motion controllers for input. The study found the game that used motion controllers to be more fun. However this could be due to the game not necessarily the controller itself. Also the study only used 18 participants all of which were computing or computer science which may have led to invalid results. However they stated that the results of the study were unclear and that more research is necessary. 


The VR interface can be evaluated using a heuristics analysis \cite{Nielsen, Pinelle}.The analysis can be used identify weaknesses in the interface and improve on them. Pinelle \textit{et al} created a series of heuristics specific to game usability. These heuristics can be used to analysis a VR interface. However Pinelle's heuristics were not designed for use in VR games some heuristics may not apply or need to be altered.  Sutcliffe presented a method based on Nielson's heuristics to analyse virtual environments \cite{sutcliffe2004heuristic}.

\section{Player and Avatar Relationship}
Won \textit{et al} conducted a series of experiments on the psychological and physiological effect VR avatars on the player \cite{won2015homuncular}. There are many other studies that look at  avatar embodiment and the fact that people can identify with avatars that differ from their own body \cite{Groen, Latoschik}. The results of Won \textit{et al}'s experiments showed that players could adapt to avatars that worked differently to their own body. Also they suggested that intuitive controls could enable faster adaptation and greater success. There findings match that of the Protues effect which says that people will subconsciously change their behaviour to match an avatar in a virtual environment \cite{won2015homuncular, yee2007proteus}.


\section{VR Controls}
The Frustration --- Aggression model says that aggression is caused by a person being blocked from reaching their goals \cite{dollard1939frustration}.  This can be applied to games as some factor can prevent the player from obtaining their in game goal.  Przybylski \textit{et al}'s work showed that it is not necessarily violence in video games that causes player aggression and frustration. They suggest that it is instead competence impeding controls that cause aggression \cite{przybylski, przybylski2010motivational}. These competence impeding controls block that player from performing their desired in game action causing frustration. 


Kovarova and Maros researched using smart phones as controllers in VR games \cite{Kovarova}. They suggested that the traditional keyboard and mouse input for video games can be restrictive in a VR game as the are not intuitive in a virtual environment. However they appear to have focused more on the hardware of a smart phone that could be useful in a VR controller. They also look at using a smart phone to make controls more intuitive \cite{Kovarova}.
Another issue is that Kovarova and Maros do not look at issues such as latency between the player input and the VR program receiving it.  Meehan says that latency issues can break a player's sense of presence in the game. Therefore any potential latency issues with using the smart phone could effect the player's VR experience \cite{Meehan}.
Bauer \textit{et al} conducted a study on using smart phones for display interactions \cite{Bauer}. They found that the small screen size had a negative effect.Users managers to complete the given task even with little prior experience with a smart phone. However using a smart phone to solve problems in 3D took longer and required more practise. VR is a 3D environment which suggests users may take have some issues with using a smart phone as a controller and may take time to adapt to it. 
\section{Human factors in VR}

\cite{stanney}



	
\bibliographystyle{ieeetr}
\bibliography{comp210_3}
	
\end{document}
