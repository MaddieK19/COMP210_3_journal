% Please do not change the document class
\documentclass{scrartcl}

% Please do not change these packages
\usepackage[hidelinks]{hyperref}
\usepackage[none]{hyphenat}
\usepackage{setspace}
\doublespace

% You may add additional packages here
\usepackage{amsmath}
\usepackage{graphicx} 
\graphicspath{ {Figures/} }

% Please include a clear, concise, and descriptive title
\title{COMP 210 Research Journal}

% Please do not change the subtitle
\subtitle{COMP210}

% Please put your student number in the author field
\author{1507866}

\begin{document}
	
\maketitle
	
\abstract{}
	
\section{VR Interfaces and Evaluation}
Mentzelopoulos \textit{et al} researched hardware interfaces for VR \cite{Mentzelopoulos}. They said that a VR interface should be spontaneous and allow the player to understand it with no explanation.
One important factor that should be considered in a VR system is low latency \cite{Mentzelopoulos, Meehan}. 
 Meehan says that latency is and important factor in VR.Meehan describes high latency as an effect that can hinder or break the player's sense of presence in a virtual environment. Their study showed that in virtual environments designed to cause stress participants with low latencies had larger increases in heart rate and nausea \cite{Meehan}.
Another factor is tracking the player's position and monitoring their input \cite{Mentzelopoulos}. McArthur found that Wiimote accessories effected the player's accuracy and error rate \cite{McArthur}.
Mentzelopoulos \textit{et al}'s study involved participants playing with either XBox controllers or Razer motion controllers for input. The study found the game that used motion controllers to be more fun. However this could be due to the game not necessarily the controller itself. Also the study only used 18 participants all of which were computing or computer science which may have led to invalid results. However they stated that the results of the study were unclear and that more research is necessary. 


The VR interface can be evaluated using a heuristics analysis \cite{Nielsen, Pinelle}.The analysis can be used identify weaknesses in the interface and improve on them. Pinelle \textit{et al} created a series of heuristics specific to game usability. These heuristics can be used to analysis a VR interface. However Pinelle's heuristics were not designed for use in VR games some heuristics may not apply or need to be altered.  Sutcliffe presented a method based on Nielson's heuristics to analyse virtual environments \cite{sutcliffe2004heuristic}.


\cite{stanney}

\section{VR Controls}
The Frustration --- Aggression model says that aggression is caused by a person being blocked from reaching their goals \cite{dollard1939frustration}.  This can be applied to games as some factor can prevent the player from obtaining their in game goal.  Przybylski \textit{et al}'s work showed that it is not necessarily violence in video games that causes player aggression and frustration. They suggest that it is instead competence impeding controls that cause aggression \cite{przybylski, przybylski2010motivational}. These competence impeding controls block that player from performing their desired in game action causing frustration. 

With VR 


Controlling avatars different to own body 
-link to incompetence 
-
\cite{won2015homuncular}

	
\bibliographystyle{ieeetr}
\bibliography{comp210_3}
	
\end{document}
